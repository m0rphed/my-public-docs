\documentclass[xetex,mathserif,serif]{beamer}
\usepackage{polyglossia}
\setdefaultlanguage[babelshorthands=true]{russian}
\usepackage{minted}
\usepackage{tabu}
\usepackage{moresize}

\useoutertheme{infolines}

\usepackage{fontspec}
\setmainfont{FreeSans}
\newfontfamily{\russianfonttt}{FreeSans}

\definecolor{links}{HTML}{2A1B81}
\hypersetup{colorlinks,linkcolor=,urlcolor=links}

\setbeamertemplate{blocks}[rounded][shadow=false]

\setbeamercolor*{block title alerted}{fg=red!50!black,bg=red!20}
\setbeamercolor*{block body alerted}{fg=black,bg=red!10}

\tabulinesep=1.2mm

\title{Об учебных практиках}
\author[Юрий Литвинов]{Юрий Литвинов\\\small{\textcolor{gray}{yurii.litvinov@gmail.com}}}
\date{29.11.2019г}

\newcommand{\attribution}[1] {
\vspace{-5mm}\begin{flushright}\begin{scriptsize}\textcolor{gray}{\textcopyright\, #1}\end{scriptsize}\end{flushright}
}

\begin{document}

	\frame{\titlepage}

	\section{Общие требования}

	\begin{frame}
		\frametitle{Зачёт по учебной практике}
		\begin{itemize}
			\item Отчёт по практике
			\begin{itemize}
				\item Порядка 5-7 содержательных страниц
				\item У каждого свой, даже если работа групповая
				\item Переиспользование фрагментов текста недопустимо
			\end{itemize}
			\item Отзыв руководителя
			\begin{itemize}
				\item В виде скана с подписью
			\end{itemize}
			\item Доклад с презентацией
			\begin{itemize}
				\item На 5-7 минут, порядка 10 слайдов
				\item Может быть одним на весь проект
			\end{itemize}
		\end{itemize}
	\end{frame}

	\section{Отчёт}

	\begin{frame}
		\frametitle{Про тексты курсовых}
		\framesubtitle{Структура отчёта}
		\begin{itemize}
			\item Титульный лист (см. \url{http://se.math.spbu.ru/SE/Members/ylitvinov/semesterWorks2018_2year}, но поправить на ``Отчёт по учебной практике '')
			\item Оглавление
			\item Введение в предметную область, постановка задачи
			\item Обзор литературы и существующих решений
			\item Описание предлагаемого решения, сравнение с существующими
			\item Заключение
			\item Список источников (ГОСТ Р 7.0.5--2008)
			\item Приложения (если есть)
		\end{itemize}
	\end{frame}

	\begin{frame}
		\frametitle{Введение}
		\begin{columns}
			\begin{column}{0.6\textwidth}
				\begin{itemize}
					\item Известная информация, ``Background''
					\item Неизвестная информация, ``Gap''
					\begin{itemize}
						\item Актуальность темы
						\item Практическая значимость
					\end{itemize}
					\item Цель работы, ``Гипотеза'' 
					\item Задачи, необходимые для достижения цели, ``Подход''
				\end{itemize}
			\end{column}
			\begin{column}{0.4\textwidth}
				\begin{center}
					\includegraphics[width=\textwidth]{introductionCone.png}
				\end{center}
			\end{column}
		\end{columns}
	\end{frame}

	\begin{frame}
		\frametitle{Постановка задачи}
		\begin{itemize}
			\item Цель работы --- одно предложение
			\begin{itemize}
				\item ``Целью работы является ...''
			\end{itemize}
			\item Задачи --- 3-5 пунктов в виде списка
			\begin{itemize}
				\item Выполнить обзор существующих решений
				\item Разработать алгоритм/архитектуру
				\item Реализовать
				\item Провести апробацию/тестирование/эксперименты
			\end{itemize}
			\item Задачи должны быть специфичны
		\end{itemize}
	\end{frame}

	\begin{frame}
		\frametitle{Обзор}
		\begin{itemize}
			\item Обзор существующих решений
			\begin{itemize}
				\item Цель и фокус обзора
				\item Критерии сравнения
				\item Выводы
			\end{itemize}
			\item Обзор используемых чужих результатов
			\begin{itemize}
				\item  Всё, написанное и придуманное не вами --- в обзор
			\end{itemize}
			\item Должен соотноситься с темой и с фокусом работы
		\end{itemize}
	\end{frame}

	\begin{frame}
		\frametitle{Описание решения}
		\begin{itemize}
			\item Желательно, чтобы разделы отвечали решению задачи из списка задач во введении
			\item Аргументированное обоснование принятых решений и отказа от альтернатив
			\item Описание программной реализации, архитектура
			\item Эксперименты и апробация
			\item Выводы и обсуждение
		\end{itemize}
	\end{frame}

	\begin{frame}
		\frametitle{Заключение}
		\begin{itemize}
			\item Перечисление результатов, выносимых на защиту
			\item Должно быть согласовано с постановкой задачи (вплоть до полного её повторения)
			\item Должно быть согласовано с текстом
			\item Никаких результатов из ниоткуда
		\end{itemize}
	\end{frame}

	\begin{frame}
		\frametitle{Общие замечания}
		\begin{itemize}
			\item Каждый рисунок --- пронумерован и подписан, есть ссылка из текста
			\item На каждый элемент списка литературы ссылка из текста
			\item Никакого плагиата!
			\item Полезно сначала написать план
			\item \url{https://papeeria.com/}, \url{https://www.overleaf.com/}
		\end{itemize}
	\end{frame}

	\section{Презентация}

	\begin{frame}
		\frametitle{Презентация}
		\begin{itemize}
			\item Доклад на 5-7 минут
			\item Возможна одна презентация на несколько человек, но у каждого должен быть свой слайд с результатами
			\item Чеклист по презентации: \url{https://docs.google.com/spreadsheets/d/1LvHveX6TdbzexuACcqGPeHIEph6cm4Hd0arCRQBqODw}
			\item Презентации прошлых лет: \url{http://se.math.spbu.ru/SE/Members/ylitvinov/semesterWorks2018_2year}
		\end{itemize}
	\end{frame}

	\begin{frame}
		\frametitle{Структура презентации}
		\begin{itemize}
			\item Титульный слайд 
			\begin{itemize}
				\item Тема, автор, научник (учёная степень если есть, должность)
			\end{itemize}
			\item Введение
			\begin{itemize}
				\item Краткий рассказ про предметную область
				\item Обосновать актуальность задачи
			\end{itemize}
			\item Постановка задачи (обязательно!)
			\begin{itemize}
				\item ``Целью работы является...''
				\item Список из 3-5 задач, которые надо было решить для достижения цели
				\item ``Сделать обзор'' (чего?), ``Разработать архитектуру'', ``Реализовать'', ``Провести эксперименты''...
			\end{itemize}
		\end{itemize}
	\end{frame}

	\begin{frame}
		\frametitle{Структура презентации (2)}
		\begin{itemize}
			\item Обзор
			\begin{itemize}
				\item Существующие решения
				\item Используемые технологии
				\item Всё, что делали не вы, но что нужно для понимания работы
			\end{itemize}
			\item Описание реализации
			\begin{itemize}
				\item Архитектура (UML-диаграммы приветствуются)
				\item Особенности реализации (то, над чем пришлось подумать)
			\end{itemize}
			\item Эксперименты
			\begin{itemize}
				\item Численные измерения (нужен матстат --- матожидание, дисперсия)
				\item Подписи к осям
				\item Примеры использования
				\item Сравнение с существующими аналогами, выводы
			\end{itemize}
		\end{itemize}
	\end{frame}

	\begin{frame}
		\frametitle{Структура презентации (3)}
		\begin{itemize}
			\item Результаты
			\begin{itemize}
				\item Список того, что выносится на защиту
				\item Должно соответствовать списку задач (лучше --- полностью повторять, с заменой ``сделать'' на ``сделано'')
				\item Всё, что перечислено в результатах, должно быть отражено ранее на слайдах
				\item Не очень приветствуются неотчуждаемые результаты (типа ``изучил'')
				\item Должно быть последним слайдом
			\end{itemize}
		\end{itemize}
	\end{frame}

\end{document}
